\documentclass{report}
\usepackage[utf8]{inputenc}
\usepackage[fontsize=15pt]{fontsize}
\title{\textbf{ME5010 : Logistic Map}}


\begin{document}
\author{Shyam Sridhar\\ Nitesh Singh \\ Rudramuni TS \\ Pavan Kumar \\ Aman Gautam }
\maketitle

\section{Introduction}

\begin{center}$x_{n+1} = rx_n(1-x_n)$\end{center}
Logistic Equation is primarily know for modeling population growth of animals and it is part \textbf{Chaos Theory}, which is branch of mathematics that demonstrates how deterministic mathematical system could lead to unpredictability.

Logistic map are it's sensitivity to input parameters. Very small changes in $x_0$ and $r$ could lead to drastic changes in predictions, and thus leading to Chaos. Due to this particular nature logistic map was once used to generate array of seemingly unrelated numbers also called pseudo-random number. It could generate unpredictability from deterministic machine. Beside that it is also used in cryptography for encryption of data where encrypted data would be very sensitive to the key i.e small variation in key would not decrypt/decode the encrypted data allowing only one and one unique key to decrypt the data.

In this project our aim is to explore behaviour of logistic map. It's sensitivity to initial parameters, period doubling, unpredictability and device a pseudo-random number generator (PRNG). We will use same PRNG to encrypt an image.



\newpage
\section{Logistic Equation}

The following equaiton also called logistic equation is used to model population growth.
\begin{equation}
x_{n+1} = rx_n(1-x_n)
\end{equation}
where:

$x_n$ = population in $n$th generation,

$r$ = growth rate




\end{document}